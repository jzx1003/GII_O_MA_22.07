\apendice{Documentación de usuario}

\section{Introducción}

En este apartado se va a explicar cómo ejecutar el código para entrenar los modelos desarrollados durante el proyecto.

\section{Requisitos de usuarios}

Los requisitos que necesita cumplir el usuario son los siguientes:

\begin{itemize}
    \item 850MB de espacio en el disco duro
    \item Python 3.10.9
    \item Anaconda
\end{itemize}

Se puede encontrar información de cómo instalar Anaconda en su página web~\cite{anaconda}.

\section{Instalación}

Para instalar el código del proyecto, podemos hacerlo desde la siguiente página citada~\cite{github:repo}. Para descargar la carpeta pulsamos el botón \textit{Code} y a continuación pulsamos \textit{Download ZIP}. En la siguiente figura~\ref{fig:github_descarga} podemos ver la ubicación de los botones.

\imagen{github_descarga}{Botones para descargar el repositorio}{1}

\section{Manual del usuario}

Para ejecutar el código que entrena los modelos debemos seguir los siguientes pasos:

\begin{itemize}
    \item Abrimos el fichero \texttt{Predicciones.ipynb}.
    \item Modificamos en la celda de <<Importación y preparación del conjunto de datos>> la ruta de la variable <<path>> para que lea el fichero CSV que contenga los datos que queramos emplear.
    \item Ejecutamos todas las celdas del fichero.
\end{itemize}

A continuación, tras esperar a que se entrenen los modelos, podemos ver en las celdas de rendimiento el error devuelto por cada modelo de forma gráfica y numérica.

Si queremos modificar los modelos, se recomienda leer la documentación técnica de programación~\ref{Documentacion_programacion}, que explica más a detalle el funcionamiento y cómo modificar distintas partes del código.