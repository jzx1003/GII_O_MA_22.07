\apendice{Especificación de Requisitos}

\section{Introducción}

Al estar empleando \textit{Scrum} como metodología ágil, se van a describir a continuación las historias de usuario (requisitos) del proyecto.

\section{Objetivos generales}

Este proyecto cuenta con los siguientes objetivos técnicos:

\begin{itemize}
    \item Realizar predicciones de variables de interés agronómico en un viñedo.
    \item Usar regresión mediante redes neuronales para la predicción en series temporales.
    \item Realizar minería de datos sobre el conjunto de datos propuesto para tratarlos previamente al desarrollo de redes neuronales.
    \item Comprender el funcionamiento de los distintos modelos de predicción a desarrollar.
    \item Desarrollar múltiples modelos de predicción empleando distintos tipos de redes neuronales.
    \item Realizar un análisis comparativo de los modelos desarrollados con los modelos de otros proyectos relacionados.
\end{itemize}

\section{Catálogo de requisitos}

A partir de los objetivos descritos en el apartado anterior, podemos desarrollar las siguientes historias de usuario o \textit{User Stories} (US):

\begin{itemize}
    \item \textbf{HU-01} El usuario debe ser capaz de cargar los datos.
    \item \textbf{HU-02} El usuario debe ser capaz de preparar los datos para el entrenamiento de los modelos.
    \item \textbf{HU-03} El usuario debe ser capaz de modificar los parámetros de entrenamiento de los modelos.
    \item \textbf{HU-04} El usuario debe poder modificar los parámetros de configuración del modelo.
    \item \textbf{HU-05} El usuario debe ser capaz de entrenar un modelo.
    \item \textbf{HU-06} El usuario debe ser capaz de validar un modelo.
    \item \textbf{HU-07} El usuario debe ser capaz de visualizar los resultados.
\end{itemize}

\section{Especificación de requisitos}

\subsection{Diagrama de casos de uso}

\imagen{diagrama-cu}{Diagrama de casos de uso}{1}

\subsection{Especificaciones de los casos de uso}

% Caso de Uso 1 -> Carga de datos.
\begin{table}[p]
	\centering
	\begin{tabularx}{\linewidth}{ p{0.21\columnwidth} p{0.71\columnwidth} }
		\toprule
		\textbf{CU-1}    & \textbf{Carga de datos}\\
		\toprule
		\textbf{Versión}              & 1.0    \\
		\textbf{Requisitos asociados} & HU-01 \\
		\textbf{Descripción}          & El usuario puede cargar los datos que se van a emplear para el entrenamiento de los modelos \\
		\textbf{Precondición}         & El usuario debe disponer de un conjunto de datos \\
		\textbf{Acciones}             &
		\begin{enumerate}
			\def\labelenumi{\arabic{enumi}.}
			\tightlist
			\item El usuario carga los datos del fichero en el código.
		\end{enumerate}\\
		\textbf{Postcondición}        & Se debe haber almacenado el conjunto de datos a emplear en una variable. \\
		\textbf{Excepciones}          & El fichero tiene una extensión no aceptada. \\
		\textbf{Importancia}          & Alta \\
		\bottomrule
	\end{tabularx}
	\caption{CU-1 Carga de datos.}
\end{table}

% Caso de Uso 2 -> Preparación de datos.
\begin{table}[p]
	\centering
	\begin{tabularx}{\linewidth}{ p{0.21\columnwidth} p{0.71\columnwidth} }
		\toprule
		\textbf{CU-2}    & \textbf{Preparación de datos}\\
		\toprule
		\textbf{Versión}              & 1.0    \\
		\textbf{Requisitos asociados} & HU-02 \\
		\textbf{Descripción}          & El usuario puede modificar el conjunto de datos cargado para emplear ciertas variables en vez de todas y descartar los datos que no se quieran emplear. También debe ser capaz de normalizar, dividir el conjunto de datos en \textit{sets} y generar secciones de datos. \\
		\textbf{Precondición}         & El usuario debe disponer de un conjunto de datos. El usuario debe poder cargar datos. \\
		\textbf{Acciones}             &
		\begin{enumerate}
			\def\labelenumi{\arabic{enumi}.}
			\tightlist
			\item El usuario carga los datos.
            \item Si el usuario quiere descartar datos, descarta todos aquellos que no quiera usar para el entrenamiento.
            \item Si el usuario quiere descartar variables del conjunto de datos, descarta aquellos que no quiera emplear para el entrenamiento.
            \item El usuario normaliza los datos con la técnica deseada.
            \item El usuario divide el conjunto de datos en conjunto de entrenamiento, de validación y de test.
            \item El usuario genera secciones de datos para las predicciones de series temporales.
		\end{enumerate}\\
		\textbf{Postcondición}        & Se debe tener un conjunto de datos preparado para el entrenamiento de los modelos. \\
		\textbf{Excepciones}          & El conjunto de datos está vacío. La suma de las proporciones de los \textit{sets} creados supera el 100\%. \\
		\textbf{Importancia}          & Alta \\
		\bottomrule
	\end{tabularx}
	\caption{CU-2 Preparación de datos.}
\end{table}

% Caso de Uso 3 -> Entrenamiento y validación de modelos.
\begin{table}[p]
	\centering
	\begin{tabularx}{\linewidth}{ p{0.21\columnwidth} p{0.71\columnwidth} }
		\toprule
		\textbf{CU-3}    & \textbf{Entrenamiento y validación de modelos}\\
		\toprule
		\textbf{Versión}              & 1.0    \\
		\textbf{Requisitos asociados} & HU-03, HU-04, HU-05, HU-06 \\
		\textbf{Descripción}          & El usuario puede entrenar un modelo y validarlo a partir de un conjunto de datos preparado.\\
		\textbf{Precondición}         & El usuario debe disponer de un conjunto de datos preparado. El usuario debe disponer de un modelo para entrenar.\\
		\textbf{Acciones}             &
		\begin{enumerate}
			\def\labelenumi{\arabic{enumi}.}
			\tightlist
			\item El usuario modifica los parámetros del entrenamiento del modelo.
            \item El usuario modifica los parámetros de la configuración del modelo.
            \item El usuario entrena el modelo.
            \item El usuario valida el modelo.
		\end{enumerate}\\
		\textbf{Postcondición}        & Se debe haber obtenido un modelo entrenado y validado. Se debe haber obtenido el error de ese modelo. \\
		\textbf{Excepciones}          & El conjunto de datos está vacío. El modelo no está bien configurado.  \\
		\textbf{Importancia}          & Alta \\
		\bottomrule
	\end{tabularx}
	\caption{CU-3 Entrenamiento y validación de modelos.}
\end{table}

% Caso de Uso 4 -> Visualizar resultados.
\begin{table}[p]
	\centering
	\begin{tabularx}{\linewidth}{ p{0.21\columnwidth} p{0.71\columnwidth} }
		\toprule
		\textbf{CU-4}    & \textbf{Visualizar los resultados}\\
		\toprule
		\textbf{Versión}              & 1.0    \\
		\textbf{Requisitos asociados} & HU-07 \\
		\textbf{Descripción}          & El usuario puede visualizar el error de cada modelo en una gráfica.\\
		\textbf{Precondición}         & El usuario debe disponer de mínimo un modelo entrenado y validado.\\
		\textbf{Acciones}             &
		\begin{enumerate}
			\def\labelenumi{\arabic{enumi}.}
			\tightlist
			\item El usuario muestra el error de cada modelo en una gráfica.
		\end{enumerate}\\
		\textbf{Postcondición}        & Se debe haber obtenido una gráfica donde se represente el error de cada modelo. \\
		\textbf{Excepciones}          & No se ha entrenado ningún modelo.  \\
		\textbf{Importancia}          & Alta \\
		\bottomrule
	\end{tabularx}
	\caption{CU-4 Visualizar resultados.}
\end{table}