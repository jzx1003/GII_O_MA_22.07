\capitulo{2}{Objetivos del proyecto}


Este proyecto está orientado al desarrollo de modelos de aprendizaje automático para las predicciones de series temporales, específicamente para la predicción de datos meteorológicos de interés agronómico en un viñedo en base a datos recogidos por sensores IoT desplegados en este.

\par

Teniendo en cuenta lo anterior, podemos establecer como objetivos de este proyecto los siguientes:

\section{Objetivos técnicos}

\begin{itemize}
    \item Realizar predicciones de variables de interés agronómico en un viñedo.
    \item Usar regresión mediante redes neuronales para la predicción en series temporales.
    \item Realizar minería de datos sobre el conjunto de datos propuesto para tratarlos previamente al desarrollo de redes neuronales.
    \item Comprender el funcionamiento de los distintos modelos de predicción a desarrollar.
    \item Desarrollar múltiples modelos de predicción empleando distintos tipos de redes neuronales.
    \item Realizar un análisis comparativo de los modelos desarrollados con los modelos de otros proyectos relacionados.
    \item Emplear un sistema de control de versiones para gestionar las modificaciones realizadas sobre el código.
    \item Realizar una investigación y una evaluación de las herramientas empleadas.
\end{itemize}

\section{Objetivos personales}

\begin{itemize}
    \item Investigar y obtener conocimientos relacionados con la minería de datos y el modelado de redes neuronales orientadas a la realización de predicciones de series temporales.
    \item Reforzar los conocimientos previos al proyecto relacionados con la minería de datos y la computación neuronal.
\end{itemize}