\capitulo{4}{Técnicas y herramientas}

\section{Python}

Python~\cite{python} es un lenguaje de programación interpretado de alto nivel que hace énfasis en la legibilidad del código. Se emplea para el desarrollo de aplicaciones de toda clase. Posee una licencia de código abierto y es uno de los lenguajes de programación más populares. Es el lenguaje por defecto de los proyectos de \textit{machine learning}.

\par

Se ha empleado como lenguaje principal en el proyecto para el análisis de los datos y para la creación y entrenamiento de cada uno de los modelos empleados.

\section{Jupyter Notebook}

Jupyter Notebook~\cite{jupyter} es una IDE basada en la web que te permite crear documentos de Jupyter Notebook, que es un documento JSON que sigue un formato de celdas de E/S que pueden contener código.

\par

Se ha empleado para almacenar todo el código en menos ficheros y facilitar más su legibilidad.

\section{Visual Studio Code}

Visual Studio Code~\cite{vscode} es un editor de código fuente que incluye soporte para la depuración, control integrado de Git, resaltado de sintaxis y otras funciones. Es un software gratuito y de código abierto que permite también la creación y edición de documentos de Jupyter Notebook.

\par

Es el editor empleado para el desarrollo del código. Esa herramienta ha permitido la creación y la modificación de los documentos de Jupyter Notebook con facilidad, además de la ejecución de estos.

\section{MATLAB}

MATLAB~\cite{matlab} es una plataforma de programación y cálculo numérico utilizada para analizar datos, desarrollar algoritmos y crear modelos.

\par

Se ha empleado para poder obtener un resultado base para establecer como objetivo mínimo entrenando un modelo de predicciones de series temporales proporcionado por una \textit{Toolbox} de la plataforma, que es el \textit{Deep Learning Toolbox}.

\section{Bibliotecas de Python}

\subsection{Matplotlib}

Matplotlib~\cite{plt} es una biblioteca que permite la creación de gráficos bidimensionales a partir de datos contenidos en listas o \textit{arrays}.

\par

Se ha empleado para la visualización de distintos atributos de los datos empleados y para la visualización de los resultados de los distintos modelos entrenados.

\subsection{seaborn}

Seaborn~\cite{seaborn} es una biblioteca basada en \textbf{matplotlib} que permite la visualización de datos.

\par

Esta biblioteca se ha empleado específicamente para la visualización de las capas de los modelos empleados.

\subsection{Numpy}

Numpy~\cite{numpy} es una biblioteca que añade un gran número de funciones matemáticas y da soporte a la creación de matrices y vectores de gran tamaño y de múltiples dimensiones.

\par

Esta biblioteca nos ha permitido realizar distintas operaciones a lo largo del proyecto gracias a lo que ofrece.

\subsection{Pandas}

 Pandas~\cite{pandas} es una biblioteca que permite la manipulación y el análisis de datos. Ofrece herramientas para la manipulación de tablas numéricas y de series temporales.

\par

Nos ha permitido realizar el análisis y la validación del conjunto de datos empleado, así como su uso en los modelos de predicción.

\subsection{TensorFlow}

TensorFlow~\cite{tensorflow} es una biblioteca desarrollada por Google que permite la construcción y el entrenamiento de redes neuronales.

\par

Es la biblioteca empleada para el desarrollo y el entrenamiento de los modelos, debido a que es sencilla de usar en comparación con otras que también permiten la creación y el entrenamiento de modelos de aprendizaje automático, como PyTorch o Scikit-learn.

\section{APIs}

\subsection{OpenWeatherMap}

OpenWeatherMap~\cite{openweathermap} es una API que ofrece datos meteorológicos mediante una llamada a esta. También es capaz de proporcionar datos históricos y predicciones como Weatherbit. Ofrece una licencia gratuita temporal para estudiantes y docentes.

\par

Se ha empleado para la obtención de datos meteorológicos de hasta hace un año. Debido a que el conjunto de datos proporcionado dispone de un periodo superior a un año, se ha optado por buscar y emplear otra API.

\subsection{Weatherbit}

Weatherbit~\cite{weatherbit} es ora API que ofrece datos meteorológicos mediante una llamada a esta. Es capaz de proporcionar datos históricos y predicciones. Esta API también ofrece una licencia temporal gratuita para investigación.

\par

Se ha empleado como sustituto de OpenWeatherMap, pues nos ofrece la posibilidad de obtener datos meteorológicos históricos de hasta hace 20 años. Además nos ofrece la posibilidad de obtener predicciones de hasta 16 días en el futuro.

\section{Github}

GitHub~\cite{github:repo} es una plataforma en línea que permite el alojamiento de proyectos mediante el software de control de versiones Git.

Se ha empleado esta herramienta para la gestión de las versiones y de las modificaciones realizadas en el código del proyecto.

\section{\LaTeX}

\LaTeX ~\cite{wiki:latex} es un sistema de composición de textos, orientado a la creación de documentos de texto con alta calidad tipográfica. Es usado principalmente para la creación de artículos y libros científicos que incluyen, entre otras cosas, expresiones matemáticas. 

Se ha empleado para el desarrollo de la memoria y los anexos a través de una plataforma gratuita.

\section{Metodología ágil}

La metodología ágil~\cite{agile} es un conjunto de técnicas aplicadas en ciclos de trabajo cortos, con el objetivo de mejorar la eficiencia del proceso de entrega de un proyecto. Esta metodología tiene como objetivo entregar valor al cliente de manera más rápida.

\par

Scrum es un marco de la metodología ágil que se basa en la colaboración, la adaptabilidad y la entrega incremental. Se organiza en \textit{sprints}, que son periodos de tiempo predefinidos y fijos en los que se realizan las actividades de desarrollo. En este caso, se ha empleado Scrum para el desarrollo del proyecto y se ha establecido una duración de una semana o dos para los \textit{sprints}, aunque algunos han tenido periodos más largos.